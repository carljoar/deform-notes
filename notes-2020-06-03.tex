\documentclass[a5paper,11pt,twoside]{article}
\usepackage[lmargin=0.5in,rmargin=0.8in,bmargin=0.3in,tmargin=0.7in]{geometry}
\usepackage{amsmath,amsfonts,amssymb,amsthm}
\usepackage{color}
\usepackage{hyperref}
\hypersetup{colorlinks=true,allcolors=black}

\usepackage{fancyhdr}
\pagestyle{fancy}
\fancyhf{}
\fancyhead[RO]{\rightmark}
\fancyhead[LE]{\leftmark}
\fancyhead[RE,LO]{\thepage}
\fancyfoot[CE,CO]{}
\fancyfoot[LE,RO]{}
\renewcommand{\headrulewidth}{0pt}
\renewcommand{\footrulewidth}{0pt}


\theoremstyle{plain}
\begingroup
\newtheorem{teo}{Theorem}[section]
\newtheorem{prop}[teo]{Proposition}
\newtheorem{conjec}[teo]{Conjecture}
\newtheorem{cor}[teo]{Corollary}
\newtheorem{lem}[teo]{Lemma}
\endgroup

\newcommand{\N}{\ensuremath{\mathbb{N}}}
\newcommand{\R}{\ensuremath{\mathbb{R}}}
\newcommand{\X}{\ensuremath{\mathfrak{X}}}
\newcommand{\cD}{\ensuremath{\mathcal{D}}}
\newcommand{\Diff}{\ensuremath{\mathrm{Diff}}}
\newcommand{\Met}{\ensuremath{\mathrm{Met}}}
\newcommand{\dvol}{\ensuremath{\,\mathrm{dvol}}}
\newcommand{\id}{\operatorname{id}}

\theoremstyle{definition}
\begingroup
\newtheorem{defin}[teo]{Definition}
\newtheorem{oss}[teo]{Remark}
\newtheorem{exam}[teo]{Example}
\newtheorem{claim}{Claim}
\endgroup



\title{Gradient flows regularized by the Riemannian stress tensor}
\author{Carl-Joar Karlsson  \\ 
	\small Department of Mathematical Sciences \\
	\small Chalmers University of Technology and The University of Gothenburg\\
	\small SE-41296 Gothenburg, Sweden}

\begin{document}
\maketitle
\thispagestyle{empty}
\setcounter{page}{0}

\section[Calculus of variations]{Calculus of variations on diffeomorphisms}
We fix an open subset $\Omega$ in $\R^d$. Diffeomorphisms on $\Omega$ form a group under composition of functions. We denote this by $\Diff(\Omega)$.

If we let $U$ denote an open subset of a Banach space $V$ we may define a class of time dependent vector fields $[0,1]\times \overline{U}\to V$ that vanish on $\partial U$ and are Lipschitz with respect to the Banach space variable. Denote by $C(t)$ this Lipschitz constant. Furthermore, assume that
\[
\int_0^1\|v(r,x)\|_Vdr+\int_0^1 C(r)dr<\infty,
\]
for some fixed $x\in U$. Assume that $v$ belongs to this class of vector fields. We define the flow of $v$ as the function $(t,x)\mapsto\varphi^v(t)(x)$, where $\varphi^v(t)(x)$ is the solution to 
\[ \partial_t y=v(t,y),\quad y(0)=x. \]
The flow of $v$ is a homeomorphism of $U$ at all times. This is a consequence of Grönwall's lemma:

\begin{lem}[Grönwall's]
\label{gronwalls}
For two functions $u,\alpha:I\to \R$, where $I$ is an interval containing zero and $u$ is bounded, assume that for some integrable function $g$,
\[
u(t)\leq g(t)+\left|\int_0^t\alpha(s)u(s)ds\right|,\quad \forall t\in I.
\]
Then, for all $t\in I$,
\[
u(t)\leq g(t)+\left|\int_0^t g(s)\alpha(s)\exp[\int_0^s\alpha(r)dr]ds\right|.
\]
If $g$ is constant, then
\[
u(t)\leq g\,\exp\left[\int_0^t\alpha(s)ds\right].
\]
\end{lem}



\subsection{The analysis of LDDMM \label{sec:LDDMM}}

\begin{defin}
	A Banach space $V$ is called \textit{admissible} if it is embedded in $C_0^1(\Omega,\R^d)$, i.e., 
	\[
	\exists C>0: \ \forall v\in V\quad \|v\|_V\geq C(\|v\|_\infty+\|Dv\|_\infty).
	\]
\end{defin}

We say that a time dependent vector in $V$ belongs to the class $\X_V^p$ on $\Omega$ in the following sense:
\[
\X_V^p(\Omega)=\left\{v\in V:\ \int_0^1\|v(t)\|_V^p<\infty\right\}.
\]
The flow of $v$ defines a 1-parameter group of diffeomorphisms by the equation $\partial_t\varphi^v(t)=v\circ\varphi^v(t)$ and the following, too, is a group:
\[
G_V=\{\varphi\in \Diff(M): \varphi=\varphi^v(1)\ \mathrm{for\ some\ } v\in\X_V^1(\Omega)\}.
\]
Define a distance $d$ on $G_V$ by
\[
d(\psi,\phi)=\inf\{\|v\|_{\X_V^1}:\ \phi=\psi\circ\varphi^v(1)\}.
\]

\begin{teo}[Trouvé, 1998]
$(G_V,d)$ is a complete metric space.
\end{teo}



\begin{teo}
Let $V$ be an admissible Hilbert space embedded in the space $C_0^{p+1}(\Omega,\R^d)$ and assume that $U:G_V\to\R$ is bounded from below and continuous with respect to uniform convergence on compact sets of derivatives up to order $p$. Then
\[
\exists \varphi\in G_V:\ E(\varphi)=\min\{E(\phi):\ \phi\in G_V\}\quad \mathrm{for\ } E(\phi)=U(\phi)+d(\id,\phi)^2.
\]
\end{teo}

Continuity for uniform convergence on compact sets means that if $\varphi_n\to\varphi$ uniformly on compact subsets of $\Omega$ and the same is true for all partial derivatives up to order $p$, then $U(\varphi_n)\to U(\varphi)$. This is the reason for requiring that $V$ be embedded in $C_0^p(\Omega)$. 
The below discussion explains why $V$ needs to be embedded in $C^{p+1}_0(\Omega)$.


Define the set $\X_1^t(\Omega)$ of absolutely integrable functions from $[0,t]$ to $C_0^1(\Omega,\R^d)$. An element of $\X_1^t(\Omega)$ is a time dependent vector field $v(t,\cdot)$ such that for each $t$, $v(t):=v(t,\cdot)\in C_0^1(\Omega,\R^d)$ and
\[
\|v\|_{\X_1^t}:=\int_0^t\|v(t)\|_{1,\infty}dt<\infty,
\]
where $\|v(t)\|_{1,\infty}=\|v(t,\cdot)\|_\infty+\|Dv(t,\cdot)\|_\infty$. Since $\|v\|_{1,\infty}$ is an upper bound for the Lipschitz constant of $v$, the flow of $v$ is defined, and
%
\begin{equation}
|\varphi^v(t)(x)-\varphi^v(t)(y)|\leq |x-y|\exp\left(\int_0^t\|v(s)\|_{1,\infty}ds\right).
\end{equation}

\begin{lem}\label{pointwise}
If $v\in\X_1^T(\Omega)$ and $v^n$ is a bounded sequence in $\X_1^T(\Omega)$ which weakly converges to some element $v\in\X_1^1(\Omega)$, then, for all $t\in[0,T]$ and for all compact subsets $Q$ of $\overline{\Omega}$,
\[
\lim_{n\to\infty}\max_{x\in Q}\left|\varphi^{v^n}(t)(x)-\varphi^{v}(t)(x)\right|=0.
\]
\end{lem}
The above lemma can be generalized to higher orders of derivatives by proving the following estimate:
%
\begin{equation} 
\label{eq:estimate}
\|\varphi^v(t)-\varphi^{u}(t)\|_{p,\infty}\leq C\left(\int_0^1\|u(t)\|_{p+1,\infty}dt\right)\int_0^t\|v-u\|_{p,\infty} dt
\end{equation}
%
That is, the variation in the $(p,\infty)$-norm of diffeomorphisms is controlled by the $(p+1,\infty)$-norm of one of the vector fields and the $(p,\infty)$-norm of both of the vector fields.

\subsubsection{Variation with respect to the vector field}
Lemma~\ref{pointwise} and~(\ref{eq:estimate}) are justified in this section. We first show that the solution to the equation
\[
\partial_t W(t)=\left(Dv\circ \varphi^v(t)\right)W(t)+h\circ \varphi^v(t)
\]
is
\[
\left.\partial_\varepsilon\right|_{\varepsilon=0}\varphi^{v+\varepsilon h}(t)=\lim_{\varepsilon\to 0}\frac{1}{\varepsilon}\left(\varphi^{v+\varepsilon h}(t)-\varphi^v(t)\right).
\]
Define
\[
a^\varepsilon(t)=\frac{1}{\varepsilon}\left(\varphi^{v+\varepsilon h}(t)-\varphi^v(t)\right)-W(t).
\]
Since $\partial_t\varphi^{v+\varepsilon h}=(v+\varepsilon h)\circ\varphi^{v+\varepsilon h}(t)$, we can express $a^\varepsilon(t)$ in the following way:
%
\begin{multline*}
a^\varepsilon(t)=\int_0^t \frac{1}{\varepsilon}\left(v\circ\varphi^{v+\varepsilon h}(r)-v\circ\varphi^v(r)\right)dr+\int_0^t h\circ \varphi^{v+\varepsilon h}(r)dr \\-\int_0^t\partial_rW(r)dr
\end{multline*}
\begin{multline*}
=\int_0^t \frac{1}{\varepsilon}\left(v\circ\varphi^{v+\varepsilon h}(r)-v\circ\varphi^v(r)\right)dr+\int_0^t h\circ \varphi^{v+\varepsilon h}(r)dr \\-\int_0^t\left(\left(Dv\circ \varphi^v(r)\right)W(r)+h\circ \varphi^v(r)\right)dr
\end{multline*}
\begin{multline*}
=\int_0^t \frac{1}{\varepsilon}\left(v\circ\varphi^{v+\varepsilon h}(r)-v\circ\varphi^v(r)\right)dr\\+\int_0^t \left(h\circ \varphi^{v+\varepsilon h}(r)-h\circ \varphi^v(r)\right)dr \\-\int_0^t\left(Dv\circ \varphi^v(r)\right)W(r)dr
\end{multline*}
\begin{multline*}
=\int_0^t \frac{1}{\varepsilon}\left(v\circ\varphi^{v+\varepsilon h}(r)-v\circ\varphi^v(r)\right)dr\\+\int_0^t \left(h\circ \varphi^{v+\varepsilon h}(r)-h\circ \varphi^v(r)\right)dr \\-\int_0^t\left(Dv\circ \varphi^v(r)\right)\left(\frac{1}{\varepsilon}\left(\varphi^{v+\varepsilon h}(r)-\varphi^v(r)\right)-a^\varepsilon(r)\right)dr
\end{multline*}
\begin{multline*}
=\int_0^t\left(Dv\circ \varphi^v(r)\right)a^\varepsilon(r)dr\\+\frac{1}{\varepsilon}\int_0^t (v+\varepsilon h)\circ\left(\varphi^{v+\varepsilon h}(r)-\varphi^v(r)\right)dr \\-\frac{1}{\varepsilon}\int_0^t\left(Dv\circ \varphi^v(r)\right)\left(\varphi^{v+\varepsilon h}(r)-\varphi^v(r)\right)dr
\end{multline*}
%\begin{multline*}
%=\int_0^t\big(Dv\circ \varphi^v(r)\big)a^\varepsilon(r)dr\\+\frac{1}{\varepsilon}\int_0^t (v+\varepsilon h- Dv\circ \varphi^v(r))\circ\left(\varphi^{v+\varepsilon h}(r)-\varphi^v(r)\right)dr.
%\end{multline*}
For $\delta>0$, define $\mu(t,\delta)=\max\{|Dv(t,x)-Dv(t,y)|: |x-y|\leq \delta\}.$ We have
\[
|v(t,x)-v(t,y)-Dv(t,x)(x-y)|\leq \mu(t,|x-y|)|x-y|.
\]
Then,
%
\begin{multline*}
|a^\varepsilon(t)|\leq \left|\int_0^t\left(Dv\circ \varphi^v(r)\right)a^\varepsilon(r)dr\right|\\+\frac{1}{\varepsilon}\int_0^t \mu(t,|\varphi^{v+\varepsilon h}(r)-\varphi^v(r)|)\left|\varphi^{v+\varepsilon h}(r)-\varphi^v(r)\right|dr 
\end{multline*}
\begin{multline*}
\leq \int_0^t\|v(r)\|_{1,\infty}|a^\varepsilon(r)|dr\\+\frac{1}{\varepsilon}\int_0^t \mu(t,|\varphi^{v+\varepsilon h}(r)-\varphi^v(r)|)\left|\varphi^{v+\varepsilon h}(r)-\varphi^v(r)\right|dr 
\end{multline*}

To proceed, we show that $|\varphi^{v+\varepsilon h}(t)(x)-\varphi^v(t)(x)|=\mathcal{O}(\varepsilon)$ using Grön\-wall's lemma and the following inequality:
%
\begin{multline*}
|\varphi^{v+\varepsilon h}(t)(x)-\varphi^v(t)(x)|\\ \leq  \varepsilon\int_0^t\|h(r)\|_\infty dr+ \int_0^t\|v(r)\|_{1,\infty}\left|\varphi^{v+\varepsilon h}(r)(x)-\varphi^v(r)(x)\right|dr .
\end{multline*}
%
That is, apply Grönwall's lemma (Lemma~\ref{gronwalls}) with $$u=|\varphi^{v+\varepsilon h}(\cdot)(x)-\varphi^v(\cdot)(x)|.$$
We obtain
%
\begin{multline*}
|\varphi^{v+\varepsilon h}(t)(x)-\varphi^v(t)(x)|\leq \\  \varepsilon\int_0^t\|h(r)\|_\infty dr 
+ \varepsilon\left|\int_0^t\|v(r)\|_{1,\infty}\left(\int_0^r\|h(s)\|_\infty ds\right)e^{\int_0^r \|v(s)\|_{1,\infty}ds} dr\right|.
\end{multline*}
%
This leads to the estimate
%
\begin{equation}
\label{estimate}
|a^\varepsilon(t)|\leq \int_0^t\|v(r)\|_{1,\infty}|a^\varepsilon(r)|dr + C\int_0^t\mu(r,C\varepsilon)dr,
\end{equation}
%
where $C$ is independent of $\varepsilon$. Since $\mu(r,\delta)\to 0$ for every fixed $r$ and since $\mu(r,\delta)\leq 2\|v(r)\|_{1,\infty}$, it follows from the dominated convergence theorem that
\[
\lim_{\varepsilon\to 0}\int_0^t \mu(r,C\varepsilon)dr=0.
\]
Grönwall's lemma on~(\ref{estimate}) therefore implies that $|a^\varepsilon(t)|\to 0$ as $\varepsilon\to 0$, which means that the solution to
\[
\partial_t W(t)=\left(Dv\circ \varphi^v(t)\right)W(t)+h\circ \varphi^v(t)
\]
(for $W$ unknown) is
\[
\left.\partial_\varepsilon\varphi^{v+\varepsilon h}(t)\right|_{\varepsilon=0}.
\]
In fact, we can prove the following theorem.

\begin{teo}
Let $v,h\in \X_1^1(\Omega)$. Then, for $x\in\Omega$,
\[
\left.\partial_\varepsilon\varphi^{v+\varepsilon h}(t)\right|_{\varepsilon=0}=D\varphi^v(t)\int_0^t\left(\big(D\varphi^v(-r)\big)h\right)\circ \varphi^v(r) dr.
\]
\end{teo}

\begin{proof}
Write $W(t)=D\varphi^v(t)(x)A(t)$, where $A(0)=0$. Then
\begin{multline*}
\partial_tW(t)=\partial_t\big(D\varphi^v(t)(x)A(t)\big)=D\big(v\circ \varphi^v(t)\big)A(t)+D\varphi^v(t)(x)\partial_tA(t)\\
=\big(Dv\circ \varphi^v(t)\big)W(t)+D\varphi^v(t)(x)\partial_tA(t).
\end{multline*}
%
Compare this to the defining equation for $W(t)$, that is, to
\[
\partial_t W(t)=\left(Dv\circ \varphi^v(t)\right)W(t)+h\circ \varphi^v(t).
\]
We conclude that
\[
\partial_tA(t)=\big(D\varphi^v(t)(x)\big)^{-1}h\circ \varphi^v(t)
\]
Recall that $\varphi^v(-t):\Omega\to\Omega$ is the inverse of $\varphi^v(t)$, and
\begin{multline*}
\id_\Omega=D\left([\varphi^v(t)]^{-1}\circ\varphi^v(t)\right)=\left(D[\varphi^v(t)]^{-1}\circ \varphi^v(t)\right)D\varphi^v(t) \\
\implies \big(D\varphi^v(t)(x)\big)^{-1}=D[\varphi^v(t)]^{-1}\circ \varphi^v(t)=D[\varphi^v(-t)]\circ \varphi^v(t),
\end{multline*}
which implies that (omitting the square brackets)
\[
\partial_tA(t)=\big(D\varphi^v(-t)\circ \varphi^v(t)\big)\big(h\circ  \varphi^v(t)\big)=\big(D\varphi^v(-t)h\big)\circ \varphi^v(t)
\]
or
\[
A(t)=\int_0^t\big(D\varphi^v(-r)h\big)\circ \varphi^v(r) dr.
\]
This yields the formula in the theorem.
\end{proof}

\begin{proof}[Proof of Lemma~\ref{pointwise}]
We compute
\begin{align*}
\left|\varphi^u(t)(x)-\varphi^v(t)(x)\right| =& \left|\int_0^t u\circ \varphi^u(r)(x)-v\circ \varphi^v(r)(x)\ dr\right|\\
\leq & \left|\int_0^t u\circ \varphi^u(r)(x)-v\circ \varphi^u(r)(x)\ dr\right|\\
&\quad +\int_0^t \left|v\circ \varphi^u(r)(x)-v\circ \varphi^v(r)(x)\right| dr\\
\leq & \left|\int_0^t u\circ \varphi^u(r)(x)-v\circ \varphi^u(r)(x)\ dr\right|\\
&\quad +\int_0^t\|v(r)\|_{1,\infty} \left|\varphi^u(r)(x)-\varphi^v(r)(x)\right| dr
\end{align*}
We apply Grönwall's lemma. In the language of Grönwall's lemma, we use
\[
u(t)=\left|\varphi^u(t)(x)-\varphi^v(t)(x)\right|,\qquad \alpha(t)=\|v(t)\|_{1,\infty}
\]
and
\[
g(t)=\left|\int_0^t u\circ \varphi^u(r)(x)-v\circ \varphi^u(r)(x)\ dr\right|.
\]
Then
\[
u(t)\leq g(t)+\left|\int_0^t \alpha(s) g(s) e^{\int_0^s\alpha(r)dr}ds\right|.
\]
Moreover,
\[
g(t)\leq \int_0^t\|u(r,\cdot)-v(r,\cdot)\|_\infty dr
\]
This yields
\begin{multline*}
\left|\varphi^u(t)(x)-\varphi^v(t)(x)\right|\leq\\
\left(1+\int_0^t \|v(s)\|_{1,\infty}e^{\int_0^s\|v(r)\|_{1,\infty}dr}ds\right)\int_0^t\|u(r,\cdot)-v(r,\cdot)\|_\infty dr,
\end{multline*}
which shows that $v\mapsto\varphi^v(t)$ is Lipschitz for $\|\cdot\|_{\X_1^t}$. Next, consider the linear map
\[
u\mapsto \int_0^t u\circ \varphi^v(r)\,dr,
\]
which is continuous, because
\[
\left|\int_0^t u\circ \varphi^v(r)\,dr\right|\leq \int_0^t\|u(r)\|_{1,\infty}dr\leq \|u\|_{\X_1^t}
\]
[TODO. Younes~L., \textit{Shapes and diffeomorphisms}, 2010, p. 170.]
\end{proof}


\subsection{Gradient flows in admissible Hilbert spaces}
Whenever $V$ is an admissible Hilbert space continuously embedded in $C_0^p(\Omega,\R^d)$, the group $G_V$ is a subset of 
\[
G_{p,\infty}=\{\varphi\in\Diff(\Omega): \|\varphi-\id\|_{p,\infty}<\infty, \|\varphi^{-1}-\id\|_{p,\infty}<\infty\}.
\]

\begin{defin}
	Let $V$ be an admissible Hilbert space continuously embedded in $C_0^p(\Omega,\R^d)$. We define the Eulerian differential $dU(\phi)\in V^*$ by
	\[
	dU(\phi)(v)=\partial_\varepsilon U(\varphi^v(\varepsilon)\circ \phi)|_{\varepsilon=0},\quad\mathrm{for\ all\ } v\in V,
	\]
	and we define the $V$-gradient $\nabla^VU(\phi)\in V$ by
	\[
	\langle \nabla^VU(\phi),v\rangle =dU(\phi)(v)\quad\forall v\in V.
	\]
\end{defin}

The kernel $K$ of $V$ provides the relation $\nabla^VU(\phi)=KdU(\phi).$

\begin{defin}
A function $U:G_{p,\infty}\to \R$ is called $(p,\infty)$-Lipschitz if for all $\varphi\in G_{p,\infty}$ there exists positive constants $\varepsilon(\varphi)$ and $C(\varphi)$ such that
\begin{multline*}
\|\psi-\varphi\|_{p,\infty}<\infty\quad\mathrm{and}\quad \|\phi-\varphi\|_{p,\infty}<\infty\\
\implies |U(\psi)-U(\phi)|\leq C(\varphi)\|\psi-\phi\|_{p,\infty}
\end{multline*}
\end{defin}

The following proposition shows that we need a smoother embedding (by one order, again because of~(\ref{eq:estimate}) to define the gradient at a time dependent vector field. In short, we can let $v(0,\cdot)=v$.

\begin{prop}
\label{prop:gradient}
	Assume that $U$ is $(p,\infty)$-Lipschitz and that $V$ is an admissible Hilbert space continuously embedded in  $C_0^{p+1}(\Omega,\R^d)$. For time dependent vector fields $v$ such that
	\[
	\lim_{\varepsilon\to 0}\frac{1}{\varepsilon}\int_0^\varepsilon\|v(t,\cdot)-v(0,\cdot)\|_Vdt=0
	\]
	we have
	\[
	dU(\phi)(v(0,\cdot))=\left.\partial_\varepsilon U(\varphi^{v(0,\cdot)}(\varepsilon)\circ \phi)\right|_{\varepsilon=0}.
	\]
\end{prop}

To the gradient, we associate a gradient descent process, namely, the flow defined by
\[
\partial_t\varphi(t)=-\nabla^V U(\varphi(t)).
\]
Whenever the integral curve of this vector field is finite, we obtain a curve in $G_V$ along which $U$ decreases, since, assuming proposition~\ref{prop:gradient} applies,
\[
\partial_tU(\varphi(t))=\langle \nabla^VU(\varphi(t)),\partial_t\varphi(t)\rangle=-\|\nabla^VU(\varphi(t))\|^2_V.
\]


\subsection{Regularizing on the space of metrics}
Let $(M,g)$ be a Riemannian manifold. Introduce the distance $d$ on the space of metrics. We want to minimize
%
\begin{equation}
\label{mainfunctional}
E(\varphi)=\frac{1}{2\sigma^2}\|I\circ\varphi^{-1}-J\|_{L^2}^2+ d(\varphi_*g,g)^2,
\end{equation}
%
where $I,J:M\to\R$ are `sufficiently smooth' functions and $\varphi_*$ is the tangent map, or push-forward, of $\varphi$. 

Notice that $E\geq 0$, which means that we can consider a sequence of metrics $(h^k)_{k=1}^\infty$, generated as
\[ h^k=\varphi^k_*g,\]
such that $E(\varphi^k)\to \inf E$ as $k\to\infty$. We would like to show that a minimizer of $E$ exists on the orbit of the push-forward action on the space of metrics.

QUESTION: In which space should variations be considered? In $\Diff(M)$? Or is the variation just a variation of $t$ (that is, of $\varphi(t)$)? A flow of a vector field is given by 
%
\begin{equation} 
\label{eq:vectorgenerator}
\partial_t\varphi^v(t)=v\circ\varphi^v(t)
\end{equation}
%
and that defines an orbit $h(t)=\varphi^v_*(t)g$, but if we consider a variation on the diffeomorphisms, we are `free' from the vector field equation~(\ref{eq:vectorgenerator}).  On the other hand, the gradient provides the structure of the vector field, namely
\[
\partial_t\varphi^\xi(t)=\xi\circ\varphi^\xi(t),\quad \xi=A^{-1}J(dE).
\]

Here, $\langle p,v\cdot q\rangle=\langle J,v\rangle$ defines the momentum map $J:T^*Q\to\mathfrak g^*$.

Is a variation on the space of metrics equivalent to a variation on the group of diffeomorphisms? I suppose that would be dependent on the functional. If the functional is 'degenerate' in some way (in one of the spaces), then that is a problem.

FORMULATION ATTEMPT: For
\[
h(t)=\varphi_*(t)g\quad \exists t_0 : h(t_0)\mathrm{\ minimizes\ }E.
\]

FORMULATION ATTEMPT: The flow of $\partial_t\varphi(t)=\nabla E(\varphi)$, where $E$ is given by~(\ref{mainfunctional}), defines an orbit...

FORMULATION ATTEMPT: For a sequence of metrics $h^k$ on the orbit of the push-forward action $\Diff(M)\times \Met(M)\to\Met(M)$ there exists a metric $h\in(h^k)$ such that 
\[
E(\varphi)=\inf E,\quad \mathrm{where\ } h=\varphi_*g.
\]

The functional $E$ has an infimum on $G_V$, since it is bounded from below.
Consider a minimizing sequence $h^k=\varphi^k_*g$, that is, a sequence such that \[E(\varphi^k)\to\inf E\quad\mathrm{as}\ k\to\infty.\]
Then

\end{document}
