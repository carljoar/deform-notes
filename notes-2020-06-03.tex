\documentclass[a5paper,11pt,twoside]{article}
\usepackage[lmargin=0.5in,rmargin=0.5in,bmargin=0.3in,tmargin=0.7in]{geometry}
\usepackage{amsmath,amsfonts,amssymb,amsthm}
\usepackage{color}
\usepackage{hyperref}
\hypersetup{colorlinks=true,allcolors=black}

\usepackage{fancyhdr}
\pagestyle{fancy}
\fancyhf{}
\fancyhead[RO]{\rightmark}
\fancyhead[LE]{\leftmark}
\fancyhead[RE,LO]{\thepage}
\fancyfoot[CE,CO]{}
\fancyfoot[LE,RO]{}
\renewcommand{\headrulewidth}{0pt}
\renewcommand{\footrulewidth}{0pt}


\theoremstyle{plain}
\begingroup
\newtheorem{teo}{Theorem}[section]
\newtheorem{prop}[teo]{Proposition}
\newtheorem{cor}[teo]{Corollary}
\newtheorem{lem}[teo]{Lemma}
\endgroup

\newcommand{\N}{\ensuremath{\mathbb{N}}}
\newcommand{\R}{\ensuremath{\mathbb{R}}}
\newcommand{\X}{\ensuremath{\mathfrak{X}}}
\newcommand{\Diff}{\ensuremath{\mathrm{Diff}}}
\newcommand{\id}{\operatorname{id}}

\theoremstyle{definition}
\begingroup
\newtheorem{defin}[teo]{Definition}
\newtheorem{oss}[teo]{Remark}
\newtheorem{exam}[teo]{Example}
\newtheorem{claim}{Claim}
\endgroup



\title{Gradient flows regularized by the Riemannian stress tensor}
\author{Carl-Joar Karlsson  \\ 
	\small Department of Mathematical Sciences \\
	\small Chalmers University of Technology and The University of Gothenburg\\
	\small SE-41296 Gothenburg, Sweden}

\begin{document}
\maketitle
\thispagestyle{empty}

\section[Calculus of variations]{Calculus of variations on diffeomorphisms}
We fix an open subset $\Omega$ in $\R^d$. Diffeomorphisms on $\Omega$ form a group under composition of functions. We denote this by $\Diff(\Omega)$.

If we let $U$ denote an open subset of a Banach space $V$ we may define a class of time dependent vector fields $[0,1]\times \overline{U}\to V$ that vanish on $\partial U$ and are Lipschitz with respect to the Banach space variable. Denote by $C(t)$ this Lipschitz constant. Furthermore, assume that
\[
\int_0^1\|v(r,x)\|_Vdr+\int_0^1 C(r)dr<\infty,
\]
for some fixed $x\in U$. Assume that $v$ belongs to this class of vector fields. We define the flow of $v$ as the function $(t,x)\mapsto\varphi^v(t)(x)$, where $\varphi^v(t)(x)$ is the solution to 
\[ \partial_t y=v(t,y),\quad y(0)=x. \]
The flow of $v$ is a homeomorphism of $U$ at all times. This is a consequence of Grönwall's lemma:

\begin{lem}[Grönwall's]
\label{gronwalls}
For two functions $u,\alpha:I\to \R$, where $I$ is an interval containing zero and $u$ is bounded, assume that for some integrable function $g$,
\[
u(t)\leq g(t)+\left|\int_0^t\alpha(s)u(s)ds\right|,\quad \forall t\in I.
\]
Then, for all $t\in I$,
\[
u(t)\leq g(t)+\left|\int_0^t g(s)\alpha(s)\exp[\int_0^s\alpha(r)dr]ds\right|.
\]
If $g$ is constant, then
\[
u(t)\leq g\,\exp\left[\int_0^t\alpha(s)ds\right].
\]
\end{lem}

\subsection{The analysis of LDDMM}

\begin{defin}
	A Banach space $V$ is called \textit{admissible} if it is embedded in $C_0^1(\Omega,\R^d)$, i.e., 
	\[
	\exists C>0: \ \forall v\in V\quad \|v\|_V\geq C(\|v\|_\infty+\|Dv\|_\infty).
	\]
\end{defin}

We say that a time dependent vector in $V$ belongs to the class $\X_V^p$ on $\Omega$ in the following sense:
\[
\X_V^p(\Omega)=\left\{v\in V:\ \int_0^1\|v(t)\|_V^p<\infty\right\}.
\]
The flow of $v$ defines a 1-parameter group of diffeomorphisms by the equation $\partial_t\varphi^v(t)=v\circ\varphi^v(t)$ and 
\[
G_V=\{\varphi\in \Diff(M): \varphi=\varphi^v(1)\ \mathrm{for\ some\ } v\in\X_V^1(\Omega)\}.
\]
Define a distance $d$ on $G_V$ by
\[
d(\psi,\phi)=\inf\{\|v\|_{\X_V^1}:\ \phi=\psi\circ\varphi^v(1)\}.
\]

\begin{teo}[Trouvé]
$(G_V,d)$ is a complete metric space.
\end{teo}



\begin{teo}
Let $V$ be an admissible Hilbert space embedded in $C_0^{p+1}(\Omega,\R^d)$ and assume that $U:G_V\to\R$ is bounded from below and continuous with respect to uniform convergence on compact sets of derivatives up to order $p$. Then
\[
\exists \varphi\in G_V:\ E(\varphi)=\min\{E(\phi):\ \phi\in G_V\}\quad \mathrm{for\ } E(\phi)=U(\phi)+d(\id,\phi)^2.
\]
\end{teo}

Continuity for uniform convergence on compact sets means that if $\varphi_n\to\varphi$ uniformly on compact subsets of $\Omega$ and the same is true for all partial derivatives up to order $p$, then $U(\varphi_n)\to U(\varphi)$. This is the reason for requiring that $V$ be embedded in $C_0^p(\Omega)$. 
The below discussion explains why $V$ needs to be embedded in $C^{p+1}_0(\Omega)$.


Define the set $\X_1^1(\Omega)$ of absolutely integrable functions from $[0,1]$ to $C_0^1(\Omega,\R^d)$. An element of $\X_1^1(\Omega)$ is a time dependent vector field $v(t,\cdot)$ such that for each $t$, $v(t):=v(t,\cdot)\in C_0^1(\Omega,\R^d)$ and
\[
\|v\|_{\X_1^1}:=\int_0^1\|v(t)\|_{1,\infty}dt<\infty,
\]
where $\|v(t)\|_{1,\infty}=\|v(t,\cdot)\|_\infty+\|Dv(t,\cdot)\|_\infty$. Since $\|v\|_{1,\infty}$ is an upper bound for the Lipschitz constant of $v$, the flow of $v$ is defined, and
%
\begin{equation}
|\varphi^v(t)(x)-\varphi^v(t)(y)|\leq |x-y|\exp\left(\int_0^t\|v(s)\|_{1,\infty}ds\right).
\end{equation}

\begin{lem}
If $v\in\X_1^1(\Omega)$ and $v^n$ is a bounded sequence in $\X_1^1(\Omega)$ which weakly converges to some element $v\in\X_1^1(\Omega)$, then, for all $t$ and for all compact subsets $Q$ of $\overline{\Omega}$,
\[
\lim_{n\to\infty}\max_{x\in Q}\left|\varphi^{v^n}(t)(x)-\varphi^{v}(t)(x)\right|=0.
\]
\end{lem}
The above lemma can be generalized to higher orders of derivatives by proving the following estimate:
\[
\|\varphi^v(t)-\varphi^{u}(t)\|_{p,\infty}\leq C\left(\int_0^1\|u(t)\|_{p+1,\infty}dt\right)\int_0^t\|v-u\|_{p,\infty} dt
\]
That is, the variation in the $(p,\infty)$-norm of diffeomorphisms is controlled by the $(p+1,\infty)$-norm of one of the vector fields and the $(p,\infty)$-norm of both of the vector fields.

\subsubsection{Variation with respect to the vector field}
We show that the solution to the equation
\[
\partial_t W(t)=\left(Dv\circ \varphi^v(t)\right)W(t)+h\circ \varphi^v(t)
\]
is
\[
\left.\partial_\varepsilon\right|_{\varepsilon=0}\varphi^{v+\varepsilon h}(t)=\lim_{\varepsilon\to 0}\frac{1}{\varepsilon}\left(\varphi^{v+\varepsilon h}(t)-\varphi^v(t)\right).
\]
Define
\[
a^\varepsilon(t)=\frac{1}{\varepsilon}\left(\varphi^{v+\varepsilon h}(t)-\varphi^v(t)\right)-W(t).
\]
Since $\partial_t\varphi^{v+\varepsilon h}=(v+\varepsilon h)\circ\varphi^{v+\varepsilon h}(t)$, we can express $a^\varepsilon(t)$ in the following way:
%
\begin{multline*}
a^\varepsilon(t)=\int_0^t \frac{1}{\varepsilon}\left(v\circ\varphi^{v+\varepsilon h}(r)-v\circ\varphi^v(r)\right)dr+\int_0^t h\circ \varphi^{v+\varepsilon h}(r)dr \\-\int_0^t\partial_rW(r)dr
\end{multline*}
\begin{multline*}
=\int_0^t \frac{1}{\varepsilon}\left(v\circ\varphi^{v+\varepsilon h}(r)-v\circ\varphi^v(r)\right)dr+\int_0^t h\circ \varphi^{v+\varepsilon h}(r)dr \\-\int_0^t\left(\left(Dv\circ \varphi^v(r)\right)W(r)+h\circ \varphi^v(r)\right)dr
\end{multline*}
\begin{multline*}
=\int_0^t \frac{1}{\varepsilon}\left(v\circ\varphi^{v+\varepsilon h}(r)-v\circ\varphi^v(r)\right)dr\\+\int_0^t \left(h\circ \varphi^{v+\varepsilon h}(r)-h\circ \varphi^v(r)\right)dr \\-\int_0^t\left(Dv\circ \varphi^v(r)\right)W(r)dr
\end{multline*}
\begin{multline*}
=\int_0^t \frac{1}{\varepsilon}\left(v\circ\varphi^{v+\varepsilon h}(r)-v\circ\varphi^v(r)\right)dr\\+\int_0^t \left(h\circ \varphi^{v+\varepsilon h}(r)-h\circ \varphi^v(r)\right)dr \\-\int_0^t\left(Dv\circ \varphi^v(r)\right)\left(\frac{1}{\varepsilon}\left(\varphi^{v+\varepsilon h}(r)-\varphi^v(r)\right)-a^\varepsilon(r)\right)dr
\end{multline*}
\begin{multline*}
=\int_0^t\left(Dv\circ \varphi^v(r)\right)a^\varepsilon(r)dr\\+\frac{1}{\varepsilon}\int_0^t (v+\varepsilon h)\circ\left(\varphi^{v+\varepsilon h}(r)-\varphi^v(r)\right)dr \\-\frac{1}{\varepsilon}\int_0^t\left(Dv\circ \varphi^v(r)\right)\left(\varphi^{v+\varepsilon h}(r)-\varphi^v(r)\right)dr
\end{multline*}
%\begin{multline*}
%=\int_0^t\big(Dv\circ \varphi^v(r)\big)a^\varepsilon(r)dr\\+\frac{1}{\varepsilon}\int_0^t (v+\varepsilon h- Dv\circ \varphi^v(r))\circ\left(\varphi^{v+\varepsilon h}(r)-\varphi^v(r)\right)dr.
%\end{multline*}
For $\delta>0$, define $\mu(t,\delta)=\max\{|Dv(t,x)-Dv(t,y)|: |x-y|\leq \delta\}.$ We have
\[
|v(t,x)-v(t,y)-Dv(t,x)(x-y)|\leq \mu(t,|x-y|)|x-y|.
\]
Then,
%
\begin{multline*}
|a^\varepsilon(t)|\leq \left|\int_0^t\left(Dv\circ \varphi^v(r)\right)a^\varepsilon(r)dr\right|\\+\frac{1}{\varepsilon}\int_0^t \mu(t,|\varphi^{v+\varepsilon h}(r)-\varphi^v(r)|)\left|\varphi^{v+\varepsilon h}(r)-\varphi^v(r)\right|dr 
\end{multline*}
\begin{multline*}
\leq \int_0^t\|v(r)\|_{1,\infty}|a^\varepsilon(r)|dr\\+\frac{1}{\varepsilon}\int_0^t \mu(t,|\varphi^{v+\varepsilon h}(r)-\varphi^v(r)|)\left|\varphi^{v+\varepsilon h}(r)-\varphi^v(r)\right|dr 
\end{multline*}

To proceed, we show that $|\varphi^{v+\varepsilon h}(t)(x)-\varphi^v(t)(x)|=\mathcal{O}(\varepsilon)$ using Grönwall's lemma and the following inequality:
%
\begin{multline*}
|\varphi^{v+\varepsilon h}(t)(x)-\varphi^v(t)(x)|\\ \leq  \varepsilon\int_0^t\|h(r)\|_\infty dr+ \int_0^t\|v(r)\|_{1,\infty}\left|\varphi^{v+\varepsilon h}(r)(x)-\varphi^v(r)(x)\right|dr .
\end{multline*}
%
That is, apply Grönwall's lemma (Lemma~\ref{gronwalls}) with $$u=|\varphi^{v+\varepsilon h}(\cdot)(x)-\varphi^v(\cdot)(x)|.$$
We obtain
%
\begin{multline*}
|\varphi^{v+\varepsilon h}(t)(x)-\varphi^v(t)(x)|\leq \\  \varepsilon\int_0^t\|h(r)\|_\infty dr 
+ \varepsilon\left|\int_0^t\|v(r)\|_{1,\infty}\left(\int_0^r\|h(s)\|_\infty ds\right)e^{\int_0^r \|v(s)\|_{1,\infty}ds} dr\right|.
\end{multline*}
%
This leads to the estimate
%
\begin{equation}
\label{estimate}
|a^\varepsilon(t)|\leq \int_0^t\|v(r)\|_{1,\infty}|a^\varepsilon(r)|dr + C\int_0^t\mu(r,C\varepsilon)dr,
\end{equation}
%
where $C$ is independent of $\varepsilon$. Since $\mu(r,\delta)\to 0$ for every fixed $r$ and since $\mu(r,\delta)\leq 2\|v(r)\|_{1,\infty}$, it follows from the dominated convergence theorem that
\[
\lim_{\varepsilon\to 0}\int_0^t \mu(r,C\varepsilon)dr=0.
\]
Grönwall's lemma on~(\ref{estimate}) therefore implies that $|a^\varepsilon(t)|\to 0$ as $\varepsilon\to 0$, which means that the solution to
\[
\partial_t W(t)=\left(Dv\circ \varphi^v(t)\right)W(t)+h\circ \varphi^v(t)
\]
(for $W$ unknown) is
\[
\left.\partial_\varepsilon\varphi^{v+\varepsilon h}(t)\right|_{\varepsilon=0}.
\]
In fact, we can prove the following theorem.

\begin{teo}
Let $v,h\in \X_1^1(\Omega)$. Then, for $x\in\Omega$,
\[
\left.\partial_\varepsilon\varphi^{v+\varepsilon h}(t)\right|_{\varepsilon=0}=D\varphi^v(t)\int_0^t\left(\big(D\varphi^v(-r)\big)h\right)\circ \varphi^v(r) dr.
\]
\end{teo}

\begin{proof}
Write $W(t)=D\varphi^v(t)(x)A(t)$, where $A(0)=0$. Then
\begin{multline*}
\partial_tW(t)=\partial_t\big(D\varphi^v(t)(x)A(t)\big)=D\big(v\circ \varphi^v(t)\big)A(t)+D\varphi^v(t)(x)\partial_tA(t)\\
=\big(Dv\circ \varphi^v(t)\big)W(t)+D\varphi^v(t)(x)\partial_tA(t).
\end{multline*}
%
Compare this to the defining equation for $W(t)$, that is, to
\[
\partial_t W(t)=\left(Dv\circ \varphi^v(t)\right)W(t)+h\circ \varphi^v(t).
\]
We conclude that
\[
\partial_tA(t)=\big(D\varphi^v(t)(x)\big)^{-1}h\circ \varphi^v(t)
\]
Recall that
\begin{multline*}
\id_\Omega=D\left([\varphi^v(t)]^{-1}\circ\varphi^v(t)\right)=\left(D[\varphi^v(t)]^{-1}\circ \varphi^v(t)\right)D\varphi^v(t) \\
\implies \big(D\varphi^v(t)(x)\big)^{-1}=D[\varphi^v(t)]^{-1}\circ \varphi^v(t)=D[\varphi^v(-t)]\circ \varphi^v(t),
\end{multline*}
which implies that (omitting the square brackets)
\[
\partial_tA(t)=\big(D\varphi^v(-t)h\big)\circ \varphi^v(t)
\]
or
\[
A(t)=\int_0^t\big(D\varphi^v(-r)h\big)\circ \varphi^v(r) dr.
\]
This yields the formula in the theorem.
\end{proof}



\subsection{Gradient flows in admissible Hilbert spaces}
\begin{defin}
	Let $V$ be an admissible Hilbert space embedded in $C_0^p(\Omega,\R^d)$. We define the Eulerian differential $dU(\phi)\in V^*$ by
	\[
	dU(\phi)(v)=\partial_\varepsilon U(\varphi^v(\varepsilon)\circ \phi)|_{\varepsilon=0},\quad\mathrm{for\ all\ } v\in V,
	\]
	and we define the $V$-gradient $\nabla^VU(\phi)\in V$ by
	\[
	\langle \nabla^VU(\phi),v\rangle =dU(\phi)(v)\quad\forall v\in V.
	\]
\end{defin}

The following proposition shows that we need a smoother embedding (by one order) to define the gradient at a time dependent vector field. In short, we can let $v=v(0,\cdot)$.

\begin{prop}
	If $V$ is an admissible Hilbert space embedded in $C_0^{p+1}(\Omega,\R^d)$ then $$dU(\phi)(v(0,\cdot))=\partial_\varepsilon U(\varphi^{v(0,\cdot)}(\varepsilon)\circ \phi).$$
\end{prop}

\end{document}
