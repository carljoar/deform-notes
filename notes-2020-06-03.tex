\documentclass[a5paper,11pt,twoside]{article}
\usepackage[lmargin=0.5in,rmargin=0.5in,bmargin=0.3in,tmargin=0.7in]{geometry}
\usepackage{amsmath,amsfonts,amssymb,amsthm}
\usepackage{color}
\usepackage{hyperref}
\hypersetup{colorlinks=true,allcolors=black}

\usepackage{fancyhdr}
\pagestyle{fancy}
\fancyhf{}
\fancyhead[RO]{\rightmark}
\fancyhead[LE]{\leftmark}
\fancyhead[RE,LO]{\thepage}
\fancyfoot[CE,CO]{}
\fancyfoot[LE,RO]{}
\renewcommand{\headrulewidth}{0pt}
\renewcommand{\footrulewidth}{0pt}


\theoremstyle{plain}
\begingroup
\newtheorem{teo}{Theorem}[section]
\newtheorem{prop}[teo]{Proposition}
\newtheorem{cor}[teo]{Corollary}
\newtheorem{lem}[teo]{Lemma}
\endgroup

\newcommand{\N}{\ensuremath{\mathbb{N}}}
\newcommand{\R}{\ensuremath{\mathbb{R}}}
\newcommand{\X}{\ensuremath{\mathfrak{X}}}
\newcommand{\Diff}{\ensuremath{\mathrm{Diff}}}
\newcommand{\id}{\operatorname{id}}

\theoremstyle{definition}
\begingroup
\newtheorem{defin}[teo]{Definition}
\newtheorem{oss}[teo]{Remark}
\newtheorem{exam}[teo]{Example}
\newtheorem{claim}{Claim}
\endgroup



\title{Gradient flows regularized by the Riemannian stress tensor}
\author{Carl-Joar Karlsson  \\ 
	\small Department of Mathematical Sciences \\
	\small Chalmers University of Technology and The University of Gothenburg\\
	\small SE-41296 Gothenburg, Sweden}

\begin{document}
\maketitle
\thispagestyle{empty}

\section[Calculus of variations]{Calculus of variations on diffeomorphisms}
We fix an open subset $\Omega$ in $\R^d$. Diffeomorphisms on $\Omega$ form a group under composition of functions. We denote this by $\Diff(\Omega)$.

\subsection{LDDMM}

\begin{defin}
	A Banach space $V$ is called \textit{admissible} if it is embedded in $C_0^1(\Omega,\R^d)$, i.e., 
	\[
	\exists C>0: \ \forall v\in V\ \|v\|_V\geq C(\|v\|_\infty+\|Dv\|_\infty).
	\]
\end{defin}

We say that a time dependent vector in $V$ belongs to the class $\X_V^p$ on $\Omega$ in the following sense:
\[
\X_V^p(\Omega)=\left\{v\in V:\ \int_0^1\|v(t)\|_V^p<\infty\right\}.
\]
The flow of $v$ defines a 1-parameter group of diffeomorphisms by the equation $\partial_t\varphi^v(t)=v\circ\varphi^v(t)$ and 
\[
G_V=\{\varphi\in \Diff(M): \varphi=\varphi^v(1)\ \mathrm{for\ some\ } v\in\X_V^1(\Omega)\}.
\]
Define a distance $d$ on $G_V$ by
\[
d(\psi,\phi)=\inf\{\|v\|_{\X_V^1}:\ \phi=\psi\circ\varphi^v(1)\}.
\]

\begin{teo}[Trouvé]
$(G_V,d)$ is a complete metric space.
\end{teo}



\begin{teo}
Let $V$ be an admissible Hilbert space embedded in $C_0^{p+1}(\Omega,\R^d)$ and assume that $U:G_V\to\R$ is bounded from below and continuous with respect to uniform convergence on compact sets of derivatives up to order $p$. Then
\[
\exists \varphi\in G_V:\ E(\varphi)=\min\{E(\phi):\ \phi\in G_V\}\quad \mathrm{for\ } E(\phi)=U(\phi)+d(\id,\phi)^2.
\]
\end{teo}

Continuity for uniform convergence on compact sets means that if $\varphi_n\to\varphi$ uniformly on compact subsets of $\Omega$ and the same is true for all partial derivatives up to order $p$, then $U(\varphi_n)\to U(\varphi)$. This is the reason for requiring that $V$ be embedded in $C_0^p(\Omega)$. The below discussion explains why $V$ needs to be embedded in $C^{p+1}_0(\Omega)$.

Define the set $\X_1^1(\Omega)$ of absolutely integrable functions from $[0,1]$ to $C_0^1(\Omega,\R^d)$. An element of $\X_1^1(\Omega)$ is a time dependent vector field $v(t,\cdot)$ such that for each $t$, $v(t):=v(t,\cdot)\in C_0^1(\Omega,\R^d)$ and
\[
\|v\|_{\X_1^1}:=\int_0^1\|v(t)\|_{1,\infty}dt<\infty,
\]
where $\|v(t)\|_{1,\infty}=\|v\|_\infty+\|Dv\|_\infty$.

\begin{lem}
If $v\in\X_1^1(\Omega)$ and $v^n$ is a bounded sequence in $\X_1^1(\Omega)$ which weakly converges to some element $v\in\X_1^1(\Omega)$, then, for all $t$ and for all compact subsets $Q$ of $\overline{\Omega}$,
\[
\lim_{n\to\infty}\max_{x\in Q}\left|\varphi^{v^n}(t)(x)-\varphi^{v}(t)(x)\right|=0.
\]
\end{lem}

The above lemma can be generalized to higher orders of derivatives by proving the following estimate:
\[
\|\varphi^v(t)-\varphi^{u}(t)\|_{p,\infty}\leq C\left(\int_0^1\|u(t)\|_{p+1,\infty}dt\right)\int_0^t\|v-u\|_{p,\infty} dt
\]
That is, the variation in the $(p,\infty)$-norm of diffeomorphisms is controlled by the $(p+1,\infty)$-norm of one of the vector fields and the $(p,\infty)$-norm of both of the vector fields.


\subsection{Gradient flows}
\begin{defin}
	Let $V$ be an admissible Hilbert space embedded in $C_0^p(\Omega,\R^d)$. We define the Eulerian differential $dU(\phi)\in V^*$ by
	\[
	dU(\phi)(v)=\partial_\varepsilon U(\varphi^v(\varepsilon)\circ \phi)|_{\varepsilon=0},\quad\mathrm{for\ all\ } v\in V,
	\]
	and we define the $V$-gradient $\nabla^VU(\phi)\in V$ by
	\[
	\langle \nabla^VU(\phi),v\rangle =dU(\phi)(v)\quad\forall v\in V.
	\]
\end{defin}

The following proposition shows that we need a smoother embedding (by one order) to define the gradient at a time dependent vector field. In short, we can let $v=v(0,\cdot)$.

\begin{prop}
	If $V$ is an admissible Hilbert space embedded in $C_0^{p+1}(\Omega,\R^d)$ then $$dU(\phi)(v(0,\cdot))=\partial_\varepsilon U(\varphi^{v(0,\cdot)}(\varepsilon)\circ \phi).$$
\end{prop}

\end{document}
