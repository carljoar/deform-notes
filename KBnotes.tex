\documentclass{article}
\usepackage[a5paper, margin=0.5in, top=0.6in]{geometry}
\usepackage{fancyhdr}
\pagestyle{fancy}
\usepackage{cmbright} % for digital version
\usepackage[T1]{fontenc}
\usepackage{amsmath}
\usepackage{amsfonts}
\usepackage{amssymb}
\usepackage{amsthm}
\usepackage{amsbsy}
\usepackage{parskip}
\usepackage{hyperref}
\usepackage{todonotes}
%\usepackage{tikz}
%\usepackage{pgfplots}

%\pgfplotsset{width=7cm,compat=1.16}
\hypersetup{colorlinks=true,allcolors=black}

\theoremstyle{plain}
\begingroup
\newtheorem{teo}{Theorem}[section]
\newtheorem{prop}[teo]{Proposition}
\newtheorem{cor}[teo]{Corollary}
\newtheorem{lem}[teo]{Lemma}
\endgroup

\theoremstyle{definition}
\begingroup
\newtheorem{defin}[teo]{Definition}
\newtheorem{oss}[teo]{Remark}
\newtheorem{exam}[teo]{Example}
\newtheorem{claim}{Claim}
\endgroup

\newcommand{\N}{\ensuremath{\mathbb{N}}}
\newcommand{\Z}{\ensuremath{\mathbb{Z}}}
\newcommand{\R}{\ensuremath{\mathbb{R}}}
\newcommand{\g}{\ensuremath{\textsl{g}}}
\newcommand{\Alt}{\ensuremath{\mathrm{Alt}}}
\newcommand{\Diff}{\ensuremath{\mathrm{Diff}}}
\newcommand{\Met}{\ensuremath{\mathrm{Met}}}
\newcommand{\Div}{\ensuremath{\mathrm{div}}}
\newcommand{\Vol}{\ensuremath{\mathrm{vol}}}
\newcommand{\Vect}{\ensuremath{\mathrm{Vect}}}
\newcommand{\id}{\ensuremath{\mathrm{id}}}
\newcommand{\lie}{\ensuremath{\mathcal{L}}}
\newcommand{\Ad}{\ensuremath{\mathrm{Ad}}}
\newcommand{\ad}{\ensuremath{\mathrm{ad}}}
\newcommand{\tr}{\ensuremath{\mathrm{tr}}}
\newcommand{\mf}{\ensuremath{\mathfrak}}

\title{Gradient Flows on Riemannian Manifolds}
\begin{document}
\maketitle 
\thispagestyle{empty}
\tableofcontents

\section{Intro}
Assume that we have a collection of ``shapes'' which are measurable functions defined on a compact Riemannian manifold $M$ without boundary. For instance, periodic images correspond to functions on the 2-torus $M=\R^2/\Z^2$ that maps into $\R$ if those images are gray-scale.



\section{Theory}
A \textit{smooth manifold} modeled on the vector space $E$ is a Hausdorff  topological space $M$ together with a collection of charts $(U_\alpha,h_\alpha)_{\alpha\in A}$ such that 
%
\begin{enumerate}
	\item $U_\alpha \subset M$ are open sets and $\bigcup_{\alpha\in A}U_\alpha = M$,
	\item $h_\alpha:U_\alpha\to h_\alpha(U_\alpha)\subset E$ are homeomorphisms onto open sets, and
	\item all maps $h_\beta \circ h^{-1}_\alpha$ are smooth.
\end{enumerate}
%
There are several possible choices of vector space $E$, for instance Hilbert spaces, Banach spaces and Fréchet spaces.

\textit{Fréchet spaces} are complete locally convex Hausdorff metrizable vector spaces, and a \textit{Fréchet manifold} is a Hausdorff space with a coordinate atlas taking values in a Fréchet space in such a way that all transition functions are smooth maps. Vector fields, tangent spaces, differential forms, principal bundles, and the like on a Fréchet manifold can be defined in the same way as for finite-dimensional manifolds. For example, a tangent vector at some point $p$ on a manifold $M$ is defined as an equivalence class of smooth parametrized curves $a^p:\R\to M$ such that $a^p(0)=p$. The collection of a such equivalence classes at $p$ is the tangent space at $p$, $T_pM$, and the union of the tangent spaces
\[
TM:=\bigcup_{p\in M}T_pM,
\]
the \textit{tangent bundle} of $M$, can be given the structure of a Fréchet manifold. The dual of a Fréchet space need not be a Fréchet space. Thus, we define 1-forms in the Fréchet setting as smooth maps $\alpha: TM\to\R$ such that the restriction $\alpha|_{T_pM}:T_pM\to\R$ is a smooth map. Differential forms of higher degree are defined analogously.

Vector fields act on real-valued functions on $M$ via the directional derivative:
\[
  X(f)=\left.\frac{d}{dt}\right|_{t=0} f(a(t)).
\]
Conversely, the \textit{differential} of $f$, $df$, is the map $TM\to \R$ defined by $X\mapsto X(f)$.

Formally for $\phi: U\to V$ and any $\omega\in \Omega^p(V)$ we define the \textit{pullback} $\phi^*$ as
\[
 \phi^*(\omega)_x(\xi_1,...,\xi_p)=\omega_{\phi(x)}\left(D_x\phi(\xi_1),...,D_x\phi(\xi_p)\right).
\]


\begin{defin}
	The \textit{exterior differential} $d:\Omega^k(M)\to \Omega^{k+1}(M)$ can be defined via local parametrizations $g:W\to N$ as follows: If $\omega$ is a smooth $k$-form on $M$ then
	\[
	d_{g(x)}\omega= \Alt^{k+1}\left((D_xg)^{-1}\right)\circ d_x(g^*\omega).
	\]
	Alternatively, define $d$ as the unique (for each $k=0,1,2,...$) linear map satisfying
	\begin{align*}
	(i)\qquad & df(X)=X(f),\quad f\in\Omega^0(M)\\
	(ii)\qquad & d(\omega\wedge \eta)=d\omega\wedge \eta + (-1)^k\omega\wedge d\eta,\quad \omega\in\Omega^k(M),\ \eta\in\Omega^l(M)\\
	(iii)\qquad & d\circ d=0
	\end{align*}
\end{defin}


\begin{defin}
	Let $M$ be a smooth $n$-manifold. For an $n$-form $\omega$ with compact support in the domain of a single oriented smooth coordinate chart $(U,h)$ the \textit{integral} is defined as 
	\[
	\int_M\omega = \int_{h(U)}\left(h^{-1}\right)^*\omega.
	\] 
\end{defin}

Let $M$ be a smooth manifold   modelled on a vector space $E$. A \textit{weak Riemannian metric} $(\cdot,\cdot)$ is a map $TM\times TM\to\R$ satisfying 
%
\begin{enumerate}
	\item $(\cdot,\cdot)_p$ is biliear for all $p\in M$, 
	\item $(X,X)_p\geq 0$ for all $X\in T_pM$, with equality only if $X=0$, and
	\item the map $p\mapsto (X,X)_p$ is smooth for any $X\in TM$. 
\end{enumerate}
%
A strong Riemannian metric is in addition required to induce the same topology as $T_pM$ inherits from $M$ on the inner product space $\left(T_pM,(\cdot,\cdot)_p\right)$. A (weak or strong) Riemannian manifold is the pair ($M,(\cdot,\cdot)$).

\begin{defin}
	The \textit{gradient} of a function $f:M\to \R$ is the vector field $\nabla f$ corresponding to the differential $\iota_{\nabla f}$, i.e., %$\omega_{\nabla f}^1$, i.e., 
	\[
	(\nabla f,X)=df(X)\qquad \forall X\in TM.
	\]
\end{defin}

\subsection{Lie groups}
A  group is a set $G'$ equipped with a binary operation $\cdot$ that combines any two elements to form a third element in such a way that four conditions called group axioms are satisfied, namely
%
\begin{description}
	\item[closure:] $a,b\in G' \implies a\cdot b\in G'$
	\item[associativity:] for all $a,b,c\in G',\quad a\cdot(b\cdot c)=(a\cdot b)\cdot c$
	\item[identity:] $\exists e\in G' : a\cdot e=e\cdot a=a$ for all $a\in G'$
	\item[existence of inverse:] for all $a\in G'$ there exists an element $c\in G'$ such that $a\cdot c=c\cdot a=e$
\end{description}
%
We will write $a\cdot b=ab$, omitting the $\cdot$, for simplicity.

A \textit{Lie group} is a smooth manifold $G$ such that its group operations (multiplication and inversion) are smooth maps from the product manifold $G\times G$ to $G$. Its \textit{Lie algebra} is the tangent space at the identity, often denoted $\mf g$, together with a bilinear skewsymmetric operation $[\cdot,\cdot]:\mf g \times \mf g \to\mf g$ called the \textit{Lie bracket}, which satisfies the Jacobi identity. 

Left multiplication on the Lie group, $L_g: G\ni h\mapsto gh\in G$, is a diffeomorphism, since the group operations are smooth. A vector field $\eta$ on $G$ is said to be left invariant if $(L_g)_*\eta=\eta$ for all $g\in G$, where $(L_g)_*$ is the tangent map of $L_g$.

The Lie algebra $\mf g$ can be identified with the set of left invariant vector fields on the group $G$, since for any $\xi\in\mf g$ we can associate a left invariant vector $\xi^g\in T_gG$ by translation via the tangent map, that is, $\xi^g=(L_g)_*\xi$. Let $\gamma$ be the integral curve of $\xi^g$, parameterized such that $\gamma_0=e$ and $(\gamma_t)'|_{t=0}=\xi$. Then
\[
  \gamma_t\circ \gamma_s=\gamma_{t+s}\quad \mathrm{and}\quad \gamma_{-t}=(\gamma_t)^{-1}\in G,
\]
that is, the integral curve defines a one-parameter (Abelian) group of diffeomorphisms.

\begin{oss}
	Is every diffeomorphism an element of a one parameter group of diffeomorphims? No. Diffeomorphisms form a topological group $\Diff(M)$ which in some respects behaves like an infinite-dimensional Lie group, but this topological group is not connected in general. The only diffeomorphisms which can be part of one-parameter groups are those in the identity component.
\end{oss}

\begin{oss}
	Warning 1.6 of Milnor's paper ``Remarks on infinite-dimensional Lie groups'': Regard $S^1$ as $\R/2\pi$. Fix a positive integer $n$ and some $\varepsilon$ such that $0<\varepsilon<1/n$ and define
	\[
	f: S^1\to S^1,\quad x\mapsto x+\pi/n+\varepsilon\sin^2(nx).
	\]
	Choosing $n$ large enough and $\varepsilon$ small enough, we can make $f$ arbitrarily close to the identity in the $C^\infty$ topology. It is a fun exercise to show that $f$ does not lie in any 1-parameter subgroup of $\Diff(S^1)$.
\end{oss}

\begin{oss}
	 What about a Banach manifold version of the diffeomorphism group? One can consider
	 \[
	 \Diff_{C^n}(M)=\{\varphi\in C^n(M,M): \varphi\ \mathrm{bijective},\ \varphi^{-1}\in C^n(M,M)\}.
	 \]
	 This is a Banach manifold and a topological group, but not a Lie group. The group operations are continuous, but not differentiable, see [arXiv:1807.11290].
\end{oss}


\subsubsection{Translation invariant metrics on Lie groups}

A Riemannian metric on a Lie group $G$ is called left invariant if it is preserved
by all left translations $L_g$, i.e., if the derivative of left translation carries every
vector to a vector of the same length. It is sufficient to give a left invariant metric at one point of the group, for instance the identity; then the metric can be carried to the remaining points by left translations. Right invariance is analogous.
%
%There is a bijective correspondence between left invariant metrics on $G$ and inner products on the Lie algebra $\mf g$ of $G$. 
%If $\langle \cdot ,\cdot \rangle$ is an inner product on $\mf g$, an induced inner product on $G$ is given by
%\[
%  (u,v)=\langle (dL_{g^{-1}})_g u, (dL_{g^{-1}})_g v\rangle,\qquad u,v\in T_gG.
%\]

Thus, there are as many left invariant Riemannian metrics on a group as there are Euclidean structures on the algebra. A Euclidean structure on the algebra is defined by a symmetric positive definite operator from the algebra to its dual space, $A:\mf g\to\mf g^*$,
\[
\langle A \xi,\eta\rangle = \langle A\eta,\xi\rangle,\quad \mathrm{for}\ \mathrm{all}\ \xi,\eta\in \mf g.
\]
We define a symmetric operator $A_g:TG_g\to T^*G_g$ by left translation:
\[
A_g = L^*_{g^{-1}}AL_{g^{-1}*}.
\]
Here, $L_{h*}:TG_g\to TG_{hg}$ is the induced tangent map of $L_h$ and $L^*_h$ denotes its adjoint, i.e., $\langle L_h^*\xi,\eta\rangle=\langle \xi,L_{h*}\eta\rangle$. The translation\todo{The pairing seems to be on the whole of $T^*M\times TM$ and not just on $\mf g^*\times \mf g$.} of $A$ induces a Riemannian metric by
\[
(\xi,\eta)_g=\langle A_g \xi,\eta\rangle=\langle A_g \eta,\xi\rangle=(\eta,\xi)_g,\quad \mathrm{where}\ \xi,\eta\in TG_g.
\]

\subsubsection{Group actions}

If $G$ is a group and $M$ is a set, a \textit{left action} of $G$ on $M$ is a map $G\times M\to M$, often written as $\theta: (g,p)\mapsto g.p$, that satisfies $g_1.(g_2.p)=(g_1g_2).p$ for all $g_1$ and $g_2$ in $G$ and all $p$ in $M$ and $\exists e\in G: e.p=p$ for all $p$ in $M$. For any $p\in M$, we define the \textit{orbit} of $p$ under the group action $\theta$ of $G$ as the set 
\[
   G.p = \mathrm{Image}_p(\theta)= \{g.p:g\in G\}.
\]



\begin{defin}
	A \textit{global flow} on a smooth manifold $M$ is a parameterized left action $a:\R\times M\to M$ satisfying
	\[
	  a(t,a(s,p))=a(t+s,p)
	\]
	for all $s,t\in\R$ and $p\in M$. 
\end{defin}

At any $p\in M$, we may define the curve $a^p:\R\to M$ by $a^p(t)=a(t,p)$, and for each $t\in\R$, $a_t:M\to M$ satisfies the group laws $a_s\circ a_t=a_{s+t}$. The curve $a^p$ defines a vector $X\in T_pM$ called the \textit{infinitesimal generator} of the flow $a$ by the relation $(a^p)'(0)=X$.

\begin{prop}
	Let $a:\R\times M\to M$ be a smooth global flow. The infinitesimal generator $X$ of $a$ is a smooth vector field on $M$ and each curve $a^p$ is an integral curve of $X$.
\end{prop}

The exponential map from a Lie algebra $\mf g$ to the corresponding Lie group $G$ can be defined provided the existence of flows. Given $\xi\in\mf g$ and the corresponding left invariant vector $\xi^g$. If the equation $(\gamma_t)'=\xi^g$\todo{Is the notation misleading?} with $\gamma_0=e$ admits a solution (in a neighborhood to $t=0$ containing $t=1$) the exponential map is defined as
\[
\exp: \mf g\to G,\quad \xi \mapsto \gamma_1.
\]
Conversely, if $\xi\in \mf g$ then this defines a vector field on $TM$ called the infinitesimal generator $X_M$ through
\[
X_M=\left.\frac{d}{dt}\right|_{t=0}\exp_{t\xi}.
\]
By the smoothness of $\exp$, the tangent space at $p'$ on the orbit $G.p$ can be constructed as the space $\{X_M\in T_{p'}M : \xi\in\mf g\}$. That is, for any $p\in M$, there is a diffeomorphism from the quotient space $G/G_{p}$ to the orbit $G.p$, which we view as a manifold, and there is an isomorphism $T_{p'}(G.p)\cong\{X_M\in T_{p'}M : \xi\in\mf g\}$, see Marsden and Ratiu, Introduction to Mechanics and Symmetry, 1994, Chapter 9.3.

\begin{oss}
	The exponential map of $\Diff(M)$ is in general neither locally surjective nor locally injective. [arXiv:1807.11290]
\end{oss}

A group action $\theta:G\times M\to M$ induces an \textit{infinitesimal group action}. Fix $p\in M$ and consider $\theta^p:G\to M$. The corresponding tangent map $\theta^p_*:TG\to T_pM$ is called the infinitesimal action of the Lie group $G$ on $M$. The map is smooth in $p$, so it induces an action $TG\to TM$. We will denote this action $\xi.X$. 

An action $G\times M\to M$ is said to act transitively if for any points $p,q\in M$ there exists $g\in G$ such that $g.p=q$. The action is called effective if
\[
g.p=h.p\ \mathrm{for}\ \mathrm{all}\ p\in M \implies g=h.
\]

\begin{exam}
	The action of the group of diffeomorphisms on a manifold $M$, $\Diff(M)$, is $(\phi,p)\mapsto \phi(p)$. It is both transitive and effective.
\end{exam}

\begin{teo}[Omori, 1978]
	If a connected Banach-Lie group $G$ acts smoothly, transitively and effectively on a compact manifold, then $G$ must be a finite-dimensional Lie group.
\end{teo}

Not all vector field generate global flows, but on compact manifolds, all vector fields are complete. [True in infinite dimensions?]

If $M$ is a subset of $\R^n$ and $H\in C^2(M)$, the dynamical system $\dot{x}=-\nabla H(x)$ is called the \textit{gradient system} of $H$. The gradient system defines a flow $\R\times M\to M$ called the \textit{gradient flow}. In this finite dimensional setting, the orbit of the gradient system is either an equilibrium or else it is decreasing along the integral curve, since if $t\mapsto \varphi(t)$ is a solution curve then
\[
\frac{d}{dt}H(\varphi(t))=(\nabla H(\varphi(t)),\dot{\varphi}(t))=-\|\nabla H(\varphi(t))\|^2\leq 0.
\]




%If $G$ acts on $M$ from the left, the action map induces a right invariant Riemannian structure on $G.p$. To understand this, we need the notion of horizontal vectors on $G$. 
Given a translation invariant metric $(\cdot,\cdot)$ on $G$,\todo{All above is left invariance, but here we want right invariance? E.g., $R_{g*}\xi$?} we may define the orthogonal\footnote{Compare this brief exposition to Ebin [1970] or Isenberg and Marsden [1982].}  complement
\[
\mathrm{Hor}_g=\mathrm{Ver}_g^\perp=\{\xi\in TG : (\xi,\eta)=0,\ \forall\eta\in\mathrm{Ver}_g\}.
\]
where $\mathrm{Ver}_g$ is the subbundle of $TG$ called the \textit{vertical distribution} associated with the action of $G$ on $M$. It is given by
\[
\mathrm{Ver}_g=\{L_{g*}\,\xi\in T_{g}G : (L_{g*}\,\xi).p=0\}.
\]
%If $G.p$ is a manifold, the mapping $\pi:G\to M$ defined by $\pi(g)=g.p$ is a submersion. Thus, $T_g\pi: \mathrm{Hor}_g\to T_{\pi(g)}G.p$ is a linear isomorphism and we may define the \textit{tangent lifted right action} of a group element $g$ on a vector field $X$ as the map \[  X\mapsto (T_g\pi)^{-1}X.\]


Add Proposition 18.4 of https://www.cis.upenn.edu/~cis610/cis610-15-sl18.pdf

Add Theorem 18.7 and Proposition 18.18 from ibid.

\begin{lem}
Assume that $G.p$ is a submanifold of $M$. Then any right invariant Riemannian metric $(\cdot,\cdot)$ on $M$ induces a Riemannian metric $r$ on $G.p$ fulfilling
\[
r_{g.p}(X.g.p,X.g.p)=(X,X)_e \qquad\forall X.g\in\mathrm{Hor}_g.
\]
\end{lem}
\begin{proof}
Since $G.p$ is a manifold, the mapping $\pi:G\to M$ defined by $\pi(g)=g.p$ is a submersion. Thus, $T_g\pi: \mathrm{Hor}_g\to T_{\pi(g)}G.p$ is a linear isomorphism. For $(x,\dot{x})\in TG.p$, take any $g$ such that $x=\pi(g)$ and define the metric $r_x$ at $x$ by
\[
r_x(\dot{x},\dot{x})=\left((T_g\pi)^{-1}\dot{x},(T_g\pi)^{-1}\dot{x}\right)_g.
\]
This is independent of the choice of $g$ by the right invariance of $(\cdot,\cdot)$.
\end{proof}

Any element $g\in G$ defines an automorphism $c_g$ of the group $G$ by conjugation, $c_g:G\to G,\ h\mapsto ghg^{-1}$. The differential of $c_g$ at the identity $e\in G$ maps the Lie algebra of $G$ to itself, that is, it defines an automorphism $\Ad_g$ on the Lie algebra. Viewed as a map of both variables, the operation $\Ad:G\times \mf g\to \mf g$ defines a left action of $G$ on its Lie algebra $\mf g$. We also see that $\Ad$ is a group homomorphism between $G$ and the group of invertible linear maps on $\mf g$, $GL(\mf g)$. This property is the reason for the name adjoint representation. The differential of the induced map $G\to\mathrm{Aut}(\mf g)$ at the identity $e\in G$ defines a map $\ad : \mf g\to \mathrm{End}(\mf g)$ called the adjoint representation of the Lie algebra $\mf g$. The relation to the Lie bracket is $[\xi,\eta]=\ad_\xi(\eta)$.

Let $M$ be a compact,\footnote{For compact manifolds M and smooth vector fields, the time-one maps of the corresponding flows, and hence the exponential maps, are well defined. Note that the group of diffeomorphisms of a noncompact manifold is not complete, and hence it is not a Lie group in our sense. (Khesin, 2009: p.13)} 
smooth $n$-dimensional manifold. Consider the
set $\Diff(M)$ of diffeomorphisms on $M$. It is an open subspace of the Fréchet
manifold of all smooth maps from $M$ to $M$. One can check that the composition
and inversion are smooth maps, so that the set $\Diff(M)$ is a Fréchet-Lie group. Its Lie algebra is given by the Lie algebra of smooth vector fields on $M$, $\Vect(M)$. The adjoint representation of $\Diff(M)$ on $\Vect(M)$ is given by coordinate changes of the vector field: for a $\phi\in\Diff(M)$, $\Ad_\phi:X\mapsto \phi_*X\circ \phi^{-1}$ and $\ad_X(Y)=\frac{\partial X}{\partial x}Y-\frac{\partial Y}{\partial x}$.



[Add page 313 and following pages from Marsden \& Ratiu, 1999. Why is $\Ad$ different from Khesin's? Because $t=0$ at page 324 in Marsden and Ratiu.]

The infinitesimal action map $TG\to T_pM$ which is the tangent map $\theta^p_*$, is linear in the vector field. Therefore, it admits the definition of a \textit{momentum map} $J$ from the cotangent bundle $T^*M$ into the dual of the Lie algebra $\mf g$ through the relation\todo{Select notation: $\xi.X$ or $X_M$?}
\[
\langle J(p,m),\xi\rangle = \langle m,X_M\rangle.
\]
Notice that the brackets denote the pairing $\mf g^*\times\mf g$ on the left hand side whereas on the right hand side, it denotes the pairing $T^*M\times TM$.

Insert remark 1 from page 387 of Marsden and Ratiu, Introduction to Mechanics and Symmetry, 1999



Recall that the dual space of a Fréchet space is not necessarily a Fréchet space. In this case, instead of considering the full dual space to an infinite dimensional Lie algebra $\mf g$, we will usually confine ourselves to considering only appropriate ``smooth duals'', the functionals from a certain $G$-invariant Fréchet subspace. (Khesin, 2009: p.20)

For a compact manifold $M$ with some fixed volume form $\Vol$, we can
define a nondegenerate $G^M$-invariant product on the current Lie algebra $\mf g\mf l(n,\R)^M$ by setting
\[
\langle X,Y\rangle = \int_M\tr(X(x),Y(x))\,\Vol(x)
\]
This inner product can be used to identify the current algebra $\mf g^M$ with a subspace in its dual $\mf g_s^*\subset \mf g^*$, where the space $\mf g_s^*$ is the smooth part of $\mf g^*$. Thanks to the nondegenerate pairing, the smooth part of the coadjoint representation of $G^M$ is isomorphic to the adjoint representation.




\subsection{The space of Riemannian metrics}
Denote by $S^2T^*M$ the bundle of  symmetric tensors of type $(0,2)$. If $C^k(S^2T^*M)$ denotes the space of $C^k$ sections of  $S^2T^*M$ with the $C^k$-topology, then
\[
C^\infty (S^2T^*M) = \bigcap_{k=0}^\infty C^k(S^2T^*M)
\]
is a Fréchet space with the $C^\infty$-topology. The set of all Riemannian metrics, $\Met(M)$, is the set of all sections in $C^\infty (S^2T^*M)$ which induce a positive definite inner product on each tangent space on the manifold $M$.
The space $\Met(M)$ is an open, convex, positive cone in $C^\infty(S^2T^*M)$, that is,
\[
a,b>0,\ \g_1,\g_2\in \Met(M) \implies a\g_1+b\g_2\in\Met(M).
\]
Therefore, $\Met$ is a Frechet manifold and since $\Met(M)$ is an open subset in the vector space $C^\infty (S^2T^*M)$  the tangent space $T_\g M$ at any $\g\in \Met(M)$ can be naturally identified with $C^\infty (S^2T^*M)$ itself.

\subsection{Parallel transport, Lie derivatives and curvature}
\todo[inline]{Introduce Parallel transport}

The covariant derivative $\nabla_X\eta$ of a tangent vector field $\eta$ in a
direction $X\in T_pM$ is the rate of change of the vector of the field $\eta$ that is parallel transported to the point $p$ along the geodesic line $\gamma$ having at this point the velocity $X$.

Suppose $X=\gamma'(0)$ is a smooth vector field on $M$. % and $\theta$ is the flow of $X$, which at $p\in M$ defines a curve $\gamma=\theta^p$. 
For any smooth vector field $Y$ on $M$ define a (rough) vector field on $M$, denoted by $\lie_XY$ and called the Lie derivative of $Y$ with respect to $X$, by
\[
(\lie_XY)_p = \lim_{t\to 0^+}\frac{1}{t}\left(d(\gamma(-t))_{\gamma(t)(p)}Y_{\gamma(t)(p)}-Y_p\right)
\]
Let, for some open interval $I$ around $0$, $\phi : M \times I \to M$ be the one-parameter semigroup of local diffeomorphisms induced by the vector flow of $Y$ and denote $\phi_t(p) := \phi(p, t)$. For each sufficiently small $t$, $\phi_t$ is a diffeomorphism from a neighborhood in $M$ to another neighborhood in $M$. The Lie derivative of $T$, denoted by $\lie_XT$, is defined at a point $p$ via push forward or pullback: 
\[
(\lie_XT)_p = \left.\frac{d}{dt}\right|_{t=0} (\phi_{-t})_*T_{\phi_t(p)}=\left.\frac{d}{dt}\right|_{t=0} (\phi_{t})^*T_{\phi_t(p)}.
\]

The Lie derivative of the one-form corresponding to a vector
field on a Riemannian manifold differs from the one-form corresponding to the
covariant derivative of the field along itself by a complete differential:
\[
\lie_X=(\nabla_X X)^\flat +\frac{1}{2}d(X,X).
\]

The (Riemannian) curvature tensor $\Omega$ describes the infinitesimal
transformation in a tangent space obtained by parallel translation around an
infinitely small parallelogram.

\todo[inline]{Introduce Lie derivative on tensors of arbitrary type.}






\section{The matching problem}
Assume throughout this section that there are two images that we want to match. We measure their ``difference'' through a function that maps those images to $\R$ such that the function attains a minimum if the two images are equal. Denote this function by $E$. In order to study deformation, we fix the images and consider the function $E$ as a function from the space of deformation mappings, which we model as a Lie group $G$, to $\R$, i.e., $E:G\to \R$. Then we have the minimization problem
\[
\min \{E(g):g\in G\}.
\]
This can often be solved by integrating $\dot{g}=-\nabla E(g)$ since this is regular in the sense that $dE/dt=(\nabla E,\dot{g})=-\|\nabla E\|\leq 0$.

In the case when $G$ is finite-dimensional, it is often enough to consider the
gradient flow without regularization. However, if G is infinite-dimensional, this might not have a solution (that is, a well defined limit). Therefore, one needs to introduce regularization to ensure that the deformation remains ``mild'' and that the flow has a limit. In this section we show how this is done without leaving the abstract framework. Let $F_{q'}$ be a functional with a (local) minimum at $q'\in Q$ and let $R:G\to \R$ be such that it attains a global minimum at $e\in G$. Then
\[
\min \{F_{q'}(g.q)+R(g):g\in G\}
\]
is a regularized problem, since any change of $g$ from the identity $e$ will increase the value of $R$. The solution is a trade-off between finding a good match, i.e., a small value of $F_{q'}$, while not being to far away from $e$.

\subsection{Analysis in the framework of LDDMM}
This section follows ``Diffeomorphic Image Matching with Left-Invariant Metrics'' by Tanya Schmah, Laurent Risser, and François-Xavier Vialard.

The flow of a time dependent vector field in $L^2([0,1],V)$ is well-defined if there exists a positive constant $C$ such that for every $v\in V$
\[
\|v\|_{1,\infty}\leq C\|v\|_V,
\]
where $\|\cdot\|_{1,\infty}$ is the Banach norm in $W^{1,\infty}(\Omega,\R^n)$. Under this hypothesis, the LDDMM problem
\[
  \min_\phi \,\frac{1}{2}\int_0^1\|v\|^2_Vdt + E(\phi(1))
\]
is well-defined and the set
\[
  G_R=\{\phi(1) : \partial_t\phi(t)=v(t)\circ\phi(t),\ v\in L^2([0,1],V),\,\phi(0)=\id\}
\]
is a group. In this setting, it has to be proven true that
\[
\partial_t\phi^{-1}(t)=-v(t)\circ \phi^{-1}(t).
\]

Similarly, the flow of $v\in C([0,1],H^s)$ defines an $H^s$-diffeomorphism provided $s>n/2+2$ where $n$ is the dimension of $M$. [Check this.]

In the following, consider $\Omega$ a closed, bounded domain and $V$ a Hilbert space of vector fields such that $v\in V$ and the spatial derivative $Dv$ vanish on the boundary. Assume there is a $C>0$ such that $\|v\|_{1,\infty}\leq C\|v\|_V$. Also, let
\[
B=C^0([0,1],C^1_\infty(\Omega,\R^n))\cap H^1([0,1],L^2(\Omega,\R^n)).
\]

\begin{lem}
Let $\phi\in B$ and denote by $\phi^{-1}$ the map $t\mapsto \phi_t^{-1}$. If $\phi_t$ is a diffeomorphism for all $t\in[0,1]$ then $\phi^{-1}\in B$.
\end{lem}

\begin{proof}
Given $\partial_t\phi_t = v(t)\circ \phi_t$ we find
\[
 0= \partial_t(\phi^{-1}_t\circ\phi_t )=\left(D\phi^{-1}_t\circ \phi_t\right) v(t)+\partial_t\phi^{-1}_t\circ\phi_t 
\]
\[
\implies \partial_t\phi^{-1}_t=-D\phi_t^{-1} \left(v(t)\circ\phi_t^{-1}\right)=-\left(D\phi_t\circ\phi_t^{-1}\right)\left(v(t)\circ\phi_t^{-1}\right)
\]
To be completed ....
\end{proof}

\begin{prop}
Solutions of $\partial_t\phi(t)=D\phi(t)\cdot v(t)$ exist in $B$. They are unique and are characterized by being solutions to 
\[
\partial_t\phi^{-1}(t)=-v(t)\circ \phi^{-1}(t).
\]
\end{prop}
This implies that the two sets
%
\begin{align*}
&\{\phi(1)\in B : \partial_t\phi(t)=D\phi(t)\cdot v(t),\ v\in L^2([0,1],V)\} \\
&\{\phi(1)\in B : \partial_t\phi(t)=v(t)\circ \phi(t),\ v\in L^2([0,1],V)\}
\end{align*}
%
coincide. When they do, we will write $G_V$ to denote any of them. Notice that on, say, $\Diff^s(M)$, the two sets do not coincide in general, since they are related via the inverse map, which might have less regularity.

\begin{teo}
If $V$ satisfies $\exists C>0: \|v\|_{1,\infty}\leq C\|v\|_V$ and $E$ is continuous w.r.t. uniform convergence of $\phi$ on every compact set in $\Omega$, then there exists a minimizer in $G_V$ of the functional 
\[
  \frac{1}{2}\int_0^1\|v\|^2_Vdt + E(\phi(1))
\]
under the constraint
\[
\partial_t\phi_t=D\phi(t)\cdot v(t).
\]
\end{teo}

For proof, see Theorem 11.2 of L. Younes, \textit{Shapes and Diffeomorphisms}.

If the spatial and convective velocities of any path $\phi_t$ are denoted by $v_R^{\phi^{-1}}$ and $v_L^\phi$, respectively, then 
\[
v_R^{\phi^{-1}}=-v_L^\phi.
\]
Thus, it follows that a right invariant metric on $G_V$ can be defined by
\[
d_R(\phi^{-1},\id)=d_L(\phi,\id),
\]
where $d_L$ is a left invariant metric on $G_V$ defined by
\[
 d_L(\phi,\id)=\inf\left\{\sqrt{\int_0^1\|v_L^\phi(t)\|_V^2 dt}: \phi_0=\id,\ \phi_1=\phi \right\}.
\]


\subsection{Met embedded in the space of symmetric tensors}
We define the inner product of two tangent vectors $S,T$, that is, two symmetric (0,2)-tensors, at the point $\g\in\Met(M)$ by
\[
(S,T)_g=\int_M \tr(g^{-1}Sg^{-1}T)\,\Vol_\g = \int_M S_{ij}T^{ij}\,\Vol_\g,
\]
where Einstein summation is understood. This inner product in invariant under the pullback action of $\Diff(M)$, that is,
\[
(\phi^*S,\phi^*T)_{\phi^*\g}=(S,T)_\g.
\]
Pullback is a right action, whereas the push forward of a diffeomorphism on a tensor in $C^\infty(S^2T^*M)$ is a left action since $(\phi\circ\varphi)_*=(\phi)_*\circ(\varphi)_*$. 

\begin{teo}
	The momentum map associated to  the co-tangent lifted (left) action  of $\Diff(M)$ on $\Met(M)$ is given by
	\[
	  J_\Met(\g,a)=2\Div (a)^\flat \otimes \Vol_\g  
	\]
	where $T^*\Met(M)$ is identified with  symmetric $(2,0)$-tensors on $M$ and  pairs with $T\Met(M)$ through
	\[
	\langle a,b\rangle = \int_M \tr (a,b)\Vol_\g=\int_M a^{ij}b_{ij}\Vol_\g.
	\]
\end{teo}

\begin{proof}
	The infinitesimal left action of $\Diff(M)$ on $\Met(M)$ is
	\[
	T\Diff\times T\Met\to T\Met,\qquad (\xi,S)\mapsto\xi.S = \left.\frac{d}{dt}\right|_{t=0} (\varphi_t)_*S = -\lie_\xi S,
	\]
	where $\xi$ generates the flow $\varphi :\R\times G \to G$. Let $g$ denote a Riemannian metric (as there is no confusion with group elements, which are denoted $\varphi$). The pairing with an element $\omega$ in $T^*\Met$ is therefore
	\[
	\langle \omega,\xi.S\rangle = -\int_M \tr (\omega,\lie_\xi S)\Vol_\g.
	\]
	The covariant divergence $\delta_g$ acting on a symmetric 2-form $\omega$ on $M$ is the first-order differential operator $(\delta_g\omega)^i=-\nabla_j\omega^{ij}$, where $\nabla$ is the Riemannian connection compatible with $\g$ and raising indices is the map $\omega\mapsto \omega^\sharp=g^{-1}\omega$. (Or $\delta_g=\tr(\xi\mapsto-\nabla_\xi\omega)$?)
	Denote
	\[
	\alpha_g:\xi\mapsto \frac{1}{2}\lie_\xi g.
	\]
	Then (Smolentsev, Journal of Mathematical Sciences, Vol. 142, No. 5, 2007),
	\[
	\langle\alpha_g(\xi),\omega\rangle=\langle\xi,\delta_g\omega\rangle,
	\]
	since integration by parts (w.r.t. Riemannian connection $\nabla^g$) yields
	%
	\begin{multline*}
	(T,\frac{1}{2}\lie_\xi g)_g=\int_M \tr(\underbrace{g^{-1}g^{-1}T}_{\alpha\in T^*\Met}\frac{1}{2}\lie_\xi g)\,\Vol_g =\\
	\left\{g\ \mathrm{symm.}\implies \xi_k=g_{jk}\xi^j=g_{kj}\xi^j\implies \frac{1}{2}(\lie_\xi g)_{ik}=\frac{1}{2}(\nabla_i\xi_k+\nabla_k\xi_i)=\nabla_k\xi_i\right\}\\
	=\int_M\tr(\alpha \nabla^g\xi)\,\Vol_g=-\int_M \nabla^g_i\alpha^{ik}\xi_k\,\Vol_g=((\delta_g \alpha)^\flat,\xi)_g.
	\end{multline*}
	%
	Here, the flat map is $\omega \mapsto\omega^\flat= g \omega$. Compare Stokes' theorem. We conclude that
	\[
	\langle \omega,\xi.g\rangle=\langle -\delta_g g^{-1}g^{-1}T,\xi\rangle,
	\]
	that is, $J(g,\omega)=-2\delta_g\omega^\flat$.
\end{proof}

\todo[inline]{Check if $$\xi.S=-\lie_\xi(\varphi_*S),\quad \varphi=\varphi_1,\ \partial_t|_{t=0}\varphi_t=\xi$$ or $$\xi.S=-\lie_\xi(\varphi_*S),\quad \varphi=\varphi_1,\ \partial_t|_{t=1}\varphi_t=\xi$$ Also check if $\lie_\xi$ and $\varphi_*$ commute because $\varphi$ is generated by $\xi$.}

\newpage
\appendix
\section{Manifolds}
\subsection{Tangent and cotangent bundles}
If $\pi:E\to M$ from a topological space $E$ into a smooth manifold $M$ has an inverse, the image $\pi^{-1}(p)$ of $p\in M$ is called a \textit{fiber} of $\pi$. It is often denoted $E_p$. A surjective smooth map $\pi: E\to M$ is said to be \textit{locally trivial} of rank $r$ if
\begin{enumerate}
	\item[(i)] each fiber is a vector space of rank $r$, and
	\item[(ii)] for all $p\in M$, there is an open neighborhood $U$ of $p$ and there exists a fiber-preserving diffeomorphism $\phi:\pi^{-1}(U)\to U\times \R^r$.
\end{enumerate}
%
The collection $\{(U,\phi)\}$ is called the \textit{local trivialization} of $E$.	A smooth \textit{vector bundle} of rank $r$ is a triple $(E,M,\pi)$, where $\pi$ is a surjective smooth map $E\to M$ which is locally trivial of rank $r$. A smooth section of $E$ is a map $S:M\to E$ such that $\pi\circ S=\id_M$. 

\begin{defin}
	The \textit{tangent bundle} of a smooth manifold is the disjoint union
	\[
	TM=\bigcup_{p\in M}\{p\}\times T_pM.
	\]
\end{defin}

For any smooth $n$-manifold $M$, the tangent bundle $TM$ has a natural topology and a smooth structure that makes it a $2n$-dimensional smooth manifold with a smooth projection $TM\to M$. Moreover, $TM$ is, with its natural topology and smooth structure, a smooth vector bundle over $M$ of rank $n$.

\begin{defin}
	The \textit{cotangent bundle} is the disjoint union of duals to $T_pM$, i.e.,
	\[
	T^*M = \bigcup_{p\in M} T_p^*M.
	\]
\end{defin}

A 1-form $\omega$ is a smooth section of $T^*M$, i.e., a map $M\to T^*M$ such that $\pi\circ \omega=\id_M$, where $\pi:T^*M\to M$ is the natural projection. A $k$-form $\omega$ is a smooth section of the exterior product $\Lambda^kT_p^*M$, that is,
\[
 \omega:M\to \Lambda^kT_p^*M = \bigcup_{p\in M}\Alt^k(T_pM)
\]
such that $\pi\circ \omega=\id_M$. Here, the alternating algebra of $T_pM$, denoted $\Alt^kT_pM$, is the collection of multilinear, alternating maps $T_pM\times ...\times T_pM\to \R$. One can show that $\Lambda^kT_p^*M$ is a vector bundle of rank $\binom{n}{k}$.

\subsection{Topology}
In the mathematical field of topology, a \textit{homeomorphism}, topological isomorphism, or bicontinuous function is a continuous function between topological spaces that has a continuous inverse function.

Let M and N be differentiable manifolds and $f: M\to N$ be a differentiable map between them. Then $f$ is called an \textit{immersion} if its derivative is everywhere injective. The map $f$ is a \textit{submersion} at a point $p\in M$ if its differential
\[
f_*:T_pM\to T_{f(p)}N
\]
is a surjective linear map. The dimension of the image $f_*(T_pM)$ is the \textit{rank} of $f$. If $f$ is an \textit{embedding}, that is, an injective immersion which is an embedding in the topological sense (i.e. homeomorphism onto its image), then $\mathrm{rank}(f)=\dim f(M)$.

Isomorphisms are defined in many different contexts but they all share a common thread. Given two objects $G$ and $H$ (which are of the same type; maybe groups, or rings, or vector spaces... etc.), an isomorphism from $G$ to $H$ is a bijection $\phi:G\to H$ which, in some sense, respects the structure of the objects. In other words, they basically identify the two objects as actually being the same object, after renaming of the elements.

For example, an isomorphism between vector spaces $V$ and $W$ is a bijection $\phi:V\to W$ which respects scalar multiplication, in that $\phi(\alpha\vec{v})=\alpha\phi(\vec{v})$ for all $\vec{v}\in V$ and $\alpha\in K$, and also respects addition in that $\phi(\vec{v}+\vec{u})=\phi(\vec{v})+\phi(\vec{u})$ for all $\vec{v},\vec{u}\in V$. (Here, we've assumed that $V$ and $W$ are both vector spaces over the same base field $K$.) Similarly, a diffeomorphism is an isomorphism of spaces equipped with a differential structure, typically differentiable manifolds.

\subsection{Integration on manifolds}
A smooth $n$-manifold is said to be \textit{orientable} is there exists an $n$-form $\omega$ such that it is nowhere vanishing. Two nowhere vanishing $n$-forms $\omega,\tau$ are said to be \textit{equivalent} if $\omega=f\tau$ for some $f\in C^\infty(M)$. An orientation on $M$ is an equivalent class of orientation forms.
\begin{lem}
On a connected, orientable smooth manifold there are precisely two orientations.
\end{lem}
A diffeomorphism $\phi$ is said to be orientation preserving if for two orientation forms $\omega$ and $\tau$, $\phi^*(\omega)=\tau$. 

\section{Hamiltonian formulation}
It turns out that the Euler–Poisson equations with quadratic Hamiltonians
have a beautiful Riemannian reformulation.
V. Arnold suggested the following general setup for the Euler equation
describing a geodesic flow on an arbitrary Lie group. Consider a (possibly
infinite-dimensional) Lie group $G$, which can be thought of as the configuration
space of some physical system.
The tangent space at the identity of the Lie group
G is the corresponding Lie algebra $\mf g$. Fix some (positive definite) quadratic
form, the energy, on the Lie algebra. We consider left (or right) translations of this quadratic
form to the tangent space at any point of the group (the ``translational symmetry''
of the energy). In this way, the energy defines a left- (respectively,
right-) invariant Riemannian metric on the group $G$. The geodesic flow on
G with respect to this energy metric represents extremals of the least-action
principle, i.e., possible motions of our physical system. To describe a geodesic
on the Lie group $G$ with an initial velocity $v(0)$, we transport its velocity vector
at any moment $t$ to the identity of the group using the left (respectively,
right) translation. This way we obtain the evolution law for $v(t)$ on the Lie
algebra $\mf g$.
To fix the notation, let $(\cdot,\cdot)$ be some left-invariant metric on the group $G$.
The geodesic flow with respect to this metric is a dynamical system on the
tangent bundle $TG$ of the group $G$. We can pull back this system to the Lie
algebra $\mf g$ of the group $G$ by left translation. That is, if $g(t)$ is a geodesic in
the group $G$ with tangent vector $g'(t)$, then the pullback $v(t)=L^*_{g(t)^{-1}}g'(t)$ is an element of the Lie algebra. Hence, the geodesic equations for $g(t)$ give us a
dynamical system
\[
\frac{d}{dt}v(t)=B(v(t))
\]
on the Lie algebra of the group $G$, where $B:\mf g\to \mf g$ is a possibly non-linear operator. This dynamical system describing the evolution of the velocity vector of a geodesic in a left-invariant metric on the Lie group $G$ is called the Euler (or Euler–Arnold) equation corresponding to this metric on $G$. It turns out that the Euler equation for a Lie group $G$ can be viewed as a Hamiltonian equation on the dual of the Lie algebra $\mf g$ in the following way. Observe that the metric $(\cdot,\cdot)_e$ at the identity $e\in G$ defines a non-degenerate bilinear form on the Lie algebra $\mf g$, and therefore, it also determines an inertia operator $A: \mf g\to \mf g^*$ such that $(v,w)_e=\langle Av,w\rangle$ for all $v,w\in \mf g$. This identification allows us to rewrite the Euler equation on the dual space $\mf g^*$. Setting $m=Av$ and $H(m)=\frac{1}{2}
\langle m,A^{-1}\rangle$ we have the following theorem, due to Arnold.

\begin{teo}
For the left invariant metric on a group  generated by the inertia operator $A:\mf g\to \mf g^*$ the Euler equation assumes the form
\[
\frac{d}{dt}m(t)=-\ad_{A^{-1}m(t)}m(t)
\]
on the dual space $\mf g^*$.
\end{teo}

Instead of using a left invariant metric on $G$, we could have
used a right invariant one. This changes the signs in the Euler equation.

The underlying reason for the Riemannian reformulation is
the fact that any geodesic problem in Riemannian geometry can be described
in terms of symplectic geometry. Geodesics on $M$ are extremals of a quadratic
Lagrangian on $TM$,
\[
L(g,v)=\frac{1}{2}(v,v)_g.
\]
They can also be described by the Hamiltonian flow on $T^*M$ for the quadratic Hamiltonian function obtained from the Lagrangian via the Legendre transform.

\section{Incompressible fluids}
Let $M$ be a compact $n$-dimensional Riemannian manifold with metric $(\cdot,\cdot)$. The group of volume-preserving diffeomorphisms of the manifold $M$ consists of all diffeomorphisms of $M$ preserving the volume form $\Vol$,
\[
S\Diff(M)=\{\phi\in\Diff(M): \phi^*\Vol=\Vol\}.
\]
We confine ourselves to the connected component of the identity diffeomorphism. For a curve $t\mapsto \gamma(t)$, where $t\in[0,1]$, in $\Diff(M)$ its $L^2$-energy is naturally given by
\[
  E(\gamma)=\frac{1}{2}\int_0^1\langle \dot{\gamma}(t),\dot{\gamma}(t)\rangle_\Diff\,dt
\]
where the (weak) Riemannian metric at each point $\gamma\in\Diff(M)$ is defined by
%
\begin{equation}
\label{eq:weakmetric}
\langle X\circ \gamma,Y\circ \gamma\rangle_\Diff = \int_M(X\circ \gamma(x),Y\circ \gamma(x))\,\Vol(x),
\end{equation}
%
where $X\circ\gamma,Y\circ\gamma\in T_\gamma\Diff(M)$.
This metric is right-invariant when restricted to the subgroup $S\Diff(M)$ of
volume-preserving diffeomorphisms, although it is not right-invariant on the
whole group $\Diff(M)$. Indeed, the change of variables in the integral
would give the Jacobian det$(\partial \eta/dx)$ as an extra factor, which, however, is identically equal to 1 for a volume-preserving map $\eta\in S\Diff(M)$. Note that for a flat manifold $M$ this metric is a flat metric on $\Diff(M)$; a neighborhood of the identity $\id\in\Diff(M)$ with the metric $\langle\cdot,\cdot\rangle_\Diff$ is isometric to a neighborhood in the pre-Hilbert space of smooth ``vector-function'' $\eta:M\to M$ with the $L^2$ inner product $\langle\cdot,\cdot\rangle_{L^2(M)}$.

\end{document}
